% outline

% Intro
% Brief history
% Why?
% goto with settrace
% goto with bytecodes
% Crashing python with bytescodes
% Problems...
% Performance!

\documentclass{beamer}
\usepackage{minted}
\usetheme{default}
\hypersetup{colorlinks=true}
\usepackage{framed}
\usepackage[normalem]{ulem}

\title{goto in Python 3. Yes. Really}
\subtitle{Kiwi PyCon 2014}
\author{Carl Cerecke}
%\institute{\href{https://github.com/cdjc/goto}}
\date{September 13-14, 2014}

\begin{document}

    \begin{frame}[plain]
        \titlepage
    \end{frame}

\begin{frame}{History}

\begin{itemize}
\item In the beginning was the \texttt{goto}
\item 1958 Heinz Zemanek expresses doubts about goto at pre-ALGOL meeting.
\item 1968 Edsgar Dijkstra ``GOTO Considered Harmful''
\item 1974 Don Knuth ``Structured Programming with go to statements''
\item 1987 Frank Rubin ` ``GOTO Considered Harmful'' Considered Harmful'
\end{itemize}

\end{frame}

\begin{frame}{Why add goto to python?}

Because it wasn't there.

Also useful for:
\begin{itemize}
\item State machines
\item Breaking out of a nested loop
\item Generating python code programmatically
\end{itemize}

\end{frame}

\begin{frame}{But it's already been done before!}

\begin{itemize}
\item April 1 2004, \href{http://entrian/goto}{entrian/goto}

\item Uses \texttt{sys.settrace}

\item Checks before the execution of \emph{every line} for goto. Slow
\end{itemize}
\end{frame}

\begin{frame}[fragile]{Goto using bytecode manipulation}

\begin{itemize}
\item Python bytecodes already have gotos.
\item JUMP\_FORWARD(delta)
\item JUMP\_ABSOLUTE(target)
\item also exotics like JUMP\_IF\_FALSE\_OR\_POP(target)
\end{itemize}

\begin{minted}{python}
    @goto
    def simple(n):
        goto .skip
        print(n)
        label .skip
\end{minted}

\end{frame}

\begin{frame}[fragile]
\begin{verbatim}
302   0 LOAD_GLOBAL      0 (goto) 
      3 LOAD_ATTR        1 (skip) 
      6 POP_TOP              
303   7 LOAD_GLOBAL      2 (print) 
     10 LOAD_FAST        0 (n) 
     13 CALL_FUNCTION    1 (1 positional, 0 keyword pair) 
     16 POP_TOP              

304  17 LOAD_GLOBAL      3 (label) 
     20 LOAD_ATTR        1 (skip) 
     23 POP_TOP              
     24 LOAD_CONST       0 (None) 
     27 RETURN_VALUE         
\end{verbatim}
\end{frame}

\begin{frame}[fragile]{Disassembly of simple function}
\begin{tabular}{l|r|l|r|l}
line & addr & opcode & par & interpretation \\
\hline
302 &          0 & LOAD\_GLOBAL         &     0 & (goto)  \\
     &           3 &  LOAD\_ATTR               &   1  & (skip)  \\
      &          6 &  POP\_TOP      &  &           \\
\hline
303   &          7 &  LOAD\_GLOBAL         &       2  & (print)  \\
          &     10 &  LOAD\_FAST           &       0 &  (n)  \\
              & 13 &  CALL\_FUNCTION        &      1  & (1 positional, 0 keyword pair)  \\
     &          16 &  POP\_TOP            &     &  \\
\hline
304   &         17 &  LOAD\_GLOBAL        &        3  & (label)  \\
     &          20 &  LOAD\_ATTR          &        1  & (skip)  \\
     &          23 &  POP\_TOP            &     &  \\
     &          24 &  LOAD\_CONST         &        0  & (None)  \\
     &          27 &  RETURN\_VALUE      &      &  \\
\end{tabular}
\end{frame}

\begin{frame}{Changes for goto}

\begin{itemize}
\item Python treats both goto and label as attribute access.

\item Want to change goto into JUMP\_ABSOLUTE

\item and label int NOP
\end{itemize}
\end{frame}

\begin{frame}[fragile]{Byte code with goto changes}
\begin{tabular}{l|r|l|r|l}
line & addr & opcode & par & interpretation \\
\hline
302 &          0 & JUMP\_ABSOLUTE         &     24 &  \\
     &           3 &  LOAD\_ATTR               &   1  & (skip)  \\
      &          6 &  POP\_TOP      &  &           \\
\hline
303   &          7 &  LOAD\_GLOBAL         &       2  & (print)  \\
          &     10 &  LOAD\_FAST           &       0 &  (n)  \\
              & 13 &  CALL\_FUNCTION        &      1  & (1 positional, 0 keyword pair)  \\
     &          16 &  POP\_TOP            &     &  \\
\hline
304   &         17 &  NOP        &          &   \\
      &         18 &  NOP        &          &   \\
      &         19 &  NOP        &          &   \\
      &         20 &  NOP        &          &   \\
      &         21 &  NOP        &          &   \\
      &         22 &  NOP        &          &   \\
      &         23 &  NOP        &          &   \\
target     &          24 &  LOAD\_CONST         &        0  & (None)  \\
     &          27 &  RETURN\_VALUE      &      &  \\
\end{tabular}
\end{frame}

\end{document}
